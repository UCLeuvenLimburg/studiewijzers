\documentclass{studiewijzer}

\studiewijzer{vgo,academiejaar=2020-2021}

\begin{document}

\showheader

\begin{tcolorbox}[title={\Large Organisatie en Begeleiding}]
    \begin{center}
        \begin{tabular}{p{\leftcolumnwidth}p{\rightcolumnwidth}}
            \toprule
            \categorylabel{Organisatie van de lessen} &
            \categorytext{
                \begin{items}
                    \item VGO: Elke week gaat er een oefensessie door van 2 uur. Daar
                          kan de student op zijn eigen tempo werken aan de inleidende labo’s,
                          of aan het eindproject. Tijdens deze oefensessies is er begeleiding
                          voorzien.
                    \item Frans: Elke week is er een oefensessie van 1u. De student
                          wordt verwacht actief deel te nemen aan de communicatieve
                          activiteiten en presentaties te houden in het Frans.
                \end{items}
            } \\
            \midrule
            \categorylabel{Verdiepingsmateriaal} &
            \categorytext{
                Zie Toledo.
            } \\
            \bottomrule
        \end{tabular}
    \end{center}
\end{tcolorbox}

\begin{tcolorbox}[title={\Large Evaluatie}]
    \begin{center}
        \begin{tabular}{p{\leftcolumnwidth}p{\rightcolumnwidth}}
            \toprule
            \categorylabel{Evaluatievormen} &
            \categorytext{
                \begin{items}
                    \item VGO: 100\% opdracht met verplichte mondelinge toelichting tijdens de examenzittijd.
                    \item Frans: 100\% opdracht
                          \begin{itemize}
                            \item Voorbereidende presentaties
                            \item Woordenschattest
                            \item Presentatie van het VGO project
                          \end{itemize}
                \end{items}
            } \\
            \midrule
            \categorylabel{Puntenverdeling evaluatieactiviteiten} &
            \categorytext{
                \begin{items}
                    \item 70\% voor het deel VGO, 30\% voor het deel Frans.
                    \item Een NA op \'e\'en van de onderdelen leidt tot een NA voor het totaal.
                    \item VGO: indien het project niet voldoet aan de basisvereisten leidt dit tot een onvoldoende.
                          Hierbij gaat het o.a. om correct ge\"implementeerde functionaliteit en de codekwaliteit.
                          De verdediging wordt multiplicatief gecombineerd met de punten op het project.
                    \item Frans: de punten worden gegeven op de voorbereidende
                          presentaties (3), op de woordenschattest (1) en op de presentatie
                          en demo van eigen VGO-project.
                \end{items}
            } \\
            \midrule
            \categorylabel{Afspraken} &
            \categorytext{
                \begin{items}
                    \item Studenten werken \emph{individueel} aan het project.
                    \item Geen enkele vorm van overleg en samenwerking tussen studenten of met derden is toegestaan.
                    \item Te laat (of niet) indienen heeft een NA op het betreffende onderdeel tot gevolg.
                \end{items}
            } \\
            \midrule
            \categorylabel{Specifieke afspraken voor de tweede examenperiode} &
            \categorytext{
                \begin{items}
                    \item VGO: Identiek aan eerste zittijd.
                    \item Frans: De student moet alle opdrachten verbeteren en opnieuw afleggen.
                \end{items}
            } \\
            \bottomrule
        \end{tabular}
    \end{center}
\end{tcolorbox}

\end{document}