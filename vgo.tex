\documentclass{studiewijzer}

\studiewijzer{vgo,academiejaar=2020-2021}

\begin{document}

\showheader

\begin{categorybox}{Organisatie en Begeleiding}
    \begin{category}{Organisatie van de lessen}
        \begin{description}
            \item[Deel VGO] Elke week gaat er een oefensessie door van 2 uur. Daar
                  kan de student op zijn eigen tempo werken aan de inleidende labo’s,
                  of aan het eindproject. Tijdens deze oefensessies is er begeleiding
                  voorzien.
            \item[Deel Frans] Elke week is er een oefensessie van 1u. De student
                  wordt verwacht actief deel te nemen aan de communicatieve
                  activiteiten en presentaties te houden in het Frans.
        \end{description}
    \end{category}

    \begin{category}{Verdiepingsmateriaal}
        Zie Toledo.
    \end{category}
\end{categorybox}

\begin{categorybox}{Evaluatie}
    \begin{category}{Evaluatievormen}
        \begin{description}
            \item[Deel VGO] 100\% opdracht met verplichte mondelinge toelichting tijdens de examenzittijd.
            \item[Deel Frans] 100\% opdracht
                    \begin{itemize}
                    \item Voorbereidende presentaties
                    \item Woordenschattest
                    \item Presentatie van het VGO project
                    \end{itemize}
        \end{description}
    \end{category}

    \begin{category}{Puntenverdeling evaluatieactiviteiten}
        \begin{description}
            \item[Puntenverdeling] 70\% voor het deel VGO, 30\% voor het deel Frans.
            \item[Niet Afgelegd] Een NA op \'e\'en van de onderdelen leidt tot een NA voor het totaal.
            \item[Deel VGO] Indien het project niet voldoet aan de basisvereisten leidt dit tot een onvoldoende.
                    Hierbij gaat het o.a. om correct ge\"implementeerde functionaliteit en de codekwaliteit.
                    De verdediging wordt multiplicatief gecombineerd met de punten op het project.
            \item[Deel Frans] De punten worden gegeven op de voorbereidende
                    presentaties~(3), op de woordenschattest~(1) en op de presentatie
                    en demo van het eigen VGO-project.
        \end{description}
    \end{category}

    \begin{category}{Afspraken}
        \begin{items}
            \item Studenten werken \emph{individueel} aan het project.
            \item Geen enkele vorm van overleg of samenwerking tussen studenten of met derden is toegestaan.
            \item Te laat (of niet) indienen heeft een NA op het betreffende onderdeel tot gevolg.
        \end{items}
    \end{category}

    \begin{category}{Deadlines}
        \begin{items}
            \item De deadline voor het inleveren van het VGO project valt op het einde van het semester.
                  Exacte datums worden via Toledo meegedeeld.
        \end{items}
    \end{category}

    \begin{category}{Tweede examenperiode}
        \begin{description}
            \item[VGO] Identiek aan eerste zittijd.
            \item[Frans] De student moet alle opdrachten verbeteren en opnieuw afleggen.
        \end{description}
    \end{category}
\end{categorybox}

\end{document}