\documentclass{studiewijzer}

\studiewijzer{bs1,academiejaar=2021-2022}

\begin{document}

\showheader

\begin{categorybox}{Organisatie en Begeleiding}
    \begin{category}{Organisatie van de lessen}
        \begin{items}
            \item Wekelijks 2 uur hoorcollege
            \item Wekelijks 2 uur practicum (behalve de laatste sessie)
            \item Laatste practicumsessie: praktische proef (3u)
                  en vindt plaats de voorlaatste of laatste week van het semester
        \end{items}
    \end{category}
\end{categorybox}

\begin{categorybox}{Evaluatie}
    \begin{category}{Evaluatievormen}
        \begin{items}
            \item Permanente evaluatie
            \item Schriftelijk examen tijdens de examenperiode
        \end{items}
    \end{category}

    \begin{category}{Puntenverdeling evaluatieactiviteiten}
        \begin{items}
            \item 50\% permanente evaluatie
            \item 50\% schriftelijk examen
        \end{items}
    \end{category}
\end{categorybox}

\begin{categorybox}{Evaluatieactiviteit Praktische Proef}
    \begin{category}{Opzet van de evaluatieactiviteit}
        Op het einde van het semester wordt een test afgenomen over de leerstof van de practica.
    \end{category}

    \begin{category}{Beoogde competenties}
        Een computersysteem leren configureren:
        \begin{items}
            \item Apparatuur
            \item Besturingsysteem
            \item Gebruikersbeheer
        \end{items}
    \end{category}

    \begin{category}{Tweede examenperiode}
        De evaluatiemethode(n) voor examenkans 2 wijkt af van die van examenkans: 100\% van de punten staan op schriftelijk examen.
    \end{category}
\end{categorybox}


\end{document}