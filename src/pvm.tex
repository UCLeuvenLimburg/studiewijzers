\documentclass{studiewijzer}

\studiewijzer{pvm,academiejaar=2021-2022}

\begin{document}

\showheader

\begin{categorybox}{Organisatie en Begeleiding}
    \begin{category}{Organisatie van de lessen}
        \begin{items}
            \item Wekelijks 2 uur practica
        \end{items}
    \end{category}
    \begin{category}{Begeleiding van de activiteiten}
        De contactmomenten zijn een combinatie van les en labo.
        Het merendeel van het werk moet gebeuren buiten de contactmomenten.
    \end{category}
    \begin{category}{Verdiepingsmateriaal}
        \begin{items}
            \item \href{https://www.stroustrup.com/4th.html}{The C++ Programming Language}
            \item \href{https://exercism.io/}{Exercism.io}
        \end{items}
    \end{category}
\end{categorybox}

\begin{categorybox}{Evaluatie}
    \begin{category}{Evaluatievorm}
        Project en examen.
    \end{category}

    \begin{category}{Puntenverdeling evaluatieactiviteiten}
        De punten van het project worden multiplicatief gecombineerd
        met de punten van het examen.
    \end{category}

    \begin{category}{Tweede examenperiode}
        De student moet zijn project verder afwerken en het examen opnieuw afleggen.
    \end{category}

    \begin{category}{Verwachtingen en afspraken}
        \begin{items}
            \item Het project moet te allen tijde beschikbaar gesteld worden via een versiebeheersysteem (Git) conform de afspraken (zie Toledo).
            \item Studenten werken individueel. Code mag niet uitgewisseld worden.
        \end{items}
    \end{category}
\end{categorybox}

\end{document}