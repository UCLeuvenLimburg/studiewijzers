\documentclass{studiewijzer}

\studiewijzer{3dcg,academiejaar=2020-2021}

\begin{document}

\showheader

\begin{categorybox}{Organisatie en Begeleiding}
    \begin{category}{Organisatie van de lessen}
        Wekelijks is er een contactmoment van 2 uur.
    \end{category}

    \begin{category}{Begeleiding van de activiteiten}
        De contactmomenten zijn een combinatie van les en labo.
        Het merendeel van het werk moet gebeuren buiten de contactmomenten.
    \end{category}

    \begin{category}{Verdiepingsmateriaal}
        Optionele lectuur:
        \begin{items}
            \item Andrew S. Glassner, An Introduction to Ray Tracing
            \item Matt Pharr, Physically Based Rendering
        \end{items}
    \end{category}
\end{categorybox}

\begin{categorybox}{Evaluatie}
    \begin{category}{Evaluatievorm}
        Project
    \end{category}

    \begin{category}{Puntenverdeling evaluatieactiviteiten}
        Studenten moeten tijdens het semester in groep een gegeven project uitbreiden.
        Studenten worden afzonderlijk beoordeeld: de prestaties van de ene student heeft geen impact op de punten van de andere teamleden.
        100\% van de punten staan op dit project, de presentatie en de verdediging ervan tijdens de examenperiode.
        Indien de student niet voldoet aan de minimumvereisten (zie Toledo) haalt deze een onvoldoende.
    \end{category}

    \begin{category}{Opzet van de evaluatieactiviteit}
        \begin{items}
            \item Het project bestaat uit het implementeren van een aantal uitbreidingen op een gegeven basisproject.
            \item Studenten moeten in groep samenwerken.
            \item Elke student wordt individueel beoordeeld op basis van zijn eigen bijdrages.
            \item Per groep kan een uitbreiding op naam staan van slechts één student.
            \item Elke uitbreiding heeft een gewicht en een categorie. Om te slagen moet een student voldoende gewicht verzameld hebben gespreid over meerdere categorieën.
            \item Tijdens de examenperiode moeten studenten in groep hun project komen verdedigen. Dit houdt in dat ze een korte presentatie moeten geven over de geïmplementeerde uitbreidingen en vragen moeten kunnen beantwoorden.
            \item Gedetailleerde informatie over het project en de mogelijke uitbreidingen is te vinden via Toledo.
        \end{items}
    \end{category}

    \begin{category}{Beoogde competenties}
        \begin{items}
            \item Werken in teamverband.
            \item Gebruik van een versiebeheersysteem.
            \item Software-engineering: C++, Testing, Design, Optimisatie.
            \item Basiselementen wiskunde voor 3D Graphics.
        \end{items}
    \end{category}

    \begin{category}{Tweede examenperiode}
        De student moet zijn project verder afwerken.
    \end{category}

    \begin{category}{Verwachtingen en afspraken}
        \begin{items}
            \item Het project moet te allen tijde beschikbaar gesteld worden via een versiebeheersysteem (Git) conform de afspraken (zie Toledo).
            \item Teams werken afzonderlijk. Code kan niet gedeeld worden tussen teams.
        \end{items}
    \end{category}

    \begin{category}{Deadlines}
        De deadline valt op het einde van het semester. Exacte datums worden via Toledo meegedeeld.
    \end{category}
\end{categorybox}

\end{document}
