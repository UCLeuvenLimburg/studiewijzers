\documentclass{studiewijzer}

\studiewijzer{3dcg,academiejaar=2020-2021}

\begin{document}

\showheader

\begin{tcolorbox}[title={\Large Organisatie en Begeleiding}]
    \begin{center}
        \begin{tabular}{p{\leftcolumnwidth}p{\rightcolumnwidth}}
            \toprule
            \categorylabel{Organisatie van de lessen} &
            \categorytext{
                Wekelijks is er een contactmoment van 2 uur.
            } \\
            \midrule
            \categorylabel{Begeleiding van de activiteiten} &
            \categorytext{
                De contactmomenten zijn een combinatie van les en labo.
                Het merendeel van het werk moet gebeuren buiten de contactmomenten.
            } \\
            \midrule
            \categorylabel{Verdiepingsmateriaal} &
            \categorytext{
                Optionele lectuur:
                \begin{itemize}
                    \item Andrew S. Glassner, An Introduction to Ray Tracing
                    \item Matt Pharr, Physically Based Rendering
                \end{itemize}
            } \\
            \bottomrule
        \end{tabular}
    \end{center}
\end{tcolorbox}

\begin{tcolorbox}[title={\Large Evaluatie}]
    \begin{center}
        \begin{tabular}{p{\leftcolumnwidth}p{\rightcolumnwidth}}
            \toprule
            \categorylabel{Evaluatievormen} &
            \categorytext{
                Project
            } \\
            \midrule
            \categorylabel{Puntenverdeling evaluatieactiviteiten} &
            \categorytext{
                100\% van de punten staan op het project, de presentatie en de verdediging ervan tijdens de examenperiode.
                Elke uitbreiding heeft een eigen gewicht en categorie.
                Om te slagen moet de student een minimum aantal uitbreidingen uitwerken gespreid over meerdere categorie\"en.
            } \\
            \bottomrule
        \end{tabular}
    \end{center}
\end{tcolorbox}

\begin{tcolorbox}[title={\Large Project}]
    \begin{tabular}{p{\leftcolumnwidth}p{\rightcolumnwidth}}
        \toprule
        \categorylabel{Opzet van de evaluatieactiviteit} &
        \categorytext{
            \begin{itemize}
                \item Studenten moeten tijdens het semester in groep een gegeven project uitbreiden.
                \item Gebruik van een versiebeheersysteem (Git) is verplicht.
                \item Er zijn een vast aantal uitbreidingen waaruit de student kan kiezen, maar eigen idee\"en zijn eveneens toegelaten na goedkeuring.
                \item Studenten moeten in groep samenwerken.
                \item Elke student wordt individueel beoordeeld op basis van zijn eigen bijdrages.
                \item Per groep kan een uitbreiding op naam staan van slechts één student.
                \item Elke uitbreiding heeft een gewicht en een categorie. Om te slagen moet een student voldoende gewicht verzameld hebben gespreid over meerdere categorieën.
                \item Tijdens de examenperiode moeten studenten in groep hun project komen verdedigen. Dit houdt in dat ze een korte presentatie moeten geven over de geïmplementeerde uitbreidingen en vragen moeten kunnen beantwoorden.
                \item Gedetailleerde informatie over het project en de mogelijke uitbreidingen is te vinden via Toledo.
            \end{itemize}
        } \\
        \midrule
        \categorylabel{Beoogde competenties} &
        \categorytext{
            \begin{itemize}
                \item Werken in teamverband
                \item Gebruik van een versiebeheer
                \item Software-engineering: C++, Testing, Design, Optimisatie, Documentatie
                \item Basiselementen wiskunde voor 3D Graphics
            \end{itemize}
        } \\
        \midrule
        \categorylabel{Specifieke afspraken voor de tweede examenperiode} &
        \categorytext{
            Teams worden door de lector heropgesteld.
            Op basis van zijn/haar prestaties in de eerste examenperiode moet de student andere (door de lector opgelegde) uitbreidingen uitwerken.
        } \\
        \midrule
        \categorylabel{Verwachtingen en afspraken} &
        \categorytext{
            \begin{itemize}
                \item Het project moet te allen tijde beschikbaar gesteld worden via een versiebeheersysteem (Git) conform de afspraken.
                \item Teams werken afzonderlijk. Code kan niet gedeeld worden tussen teams.
            \end{itemize}
        } \\
        \categorylabel{Deadlines} &
        \categorytext{
            De deadline valt op het einde van het semester. Exacte datums worden via Toledo meegedeeld.
        } \\
        \bottomrule
    \end{tabular}
\end{tcolorbox}


\end{document}