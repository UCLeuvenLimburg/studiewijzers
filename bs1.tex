\documentclass{studiewijzer}

\studiewijzer{bs1,academiejaar=2020-2021}

\begin{document}

\showheader

\section*{Organisatie en Begeleiding}

\begin{center}
    \begin{tabular}{p{\leftcolumnwidth}p{\rightcolumnwidth}}
        \toprule
        \categorylabel{Organisatie van de lessen} &
        \categorytext{
            \begin{itemize}
                \item Wekelijks 2 uur hoorcollege
                \item Wekelijks 2 uur practicum (behalve het laatste practicum)
                \item PE-test (3u) tijdens het laatste practicum (planning: de voorlaatste of laatste week van het semester)
            \end{itemize}} \\
        \bottomrule
    \end{tabular}
\end{center}

\section*{Evaluatie}

\begin{center}
    \begin{tabular}{p{\leftcolumnwidth}p{\rightcolumnwidth}}
        \toprule
        \categorylabel{Evaluatievormen} &
        \categorytext{
            \begin{itemize}
                \item Permanente evaluatie
                \item Schriftelijk examen tijdens de examenperiode
            \end{itemize}} \\
        \midrule
        \categorylabel{Puntenverdeling evaluatieactiviteiten} &
        \categorytext{
            \begin{itemize}
                \item 50\% permanente evaluatie
                \item 50\% schriftelijk examen
            \end{itemize}} \\
        \bottomrule
    \end{tabular}
\end{center}

\section*{Partiële of Permanente Evaluatie}
\subsection*{Evaluatieactiviteit Praktische Proef}

\begin{center}
    \begin{tabular}{p{\leftcolumnwidth}p{\rightcolumnwidth}}
        \toprule
        \categorylabel{Opzet van de evaluatieactiviteit} &
        \categorytext{
            Op het einde van het semester wordt een test afgenomen over de leerstof van de practica.
        } \\
        \midrule
        \categorylabel{Beoogde competenties} &
        \categorytext{
            Een computersysteem leren configureren: apparatuur, besturingsysteem, gebruikersbeheer
        } \\
        \midrule
        \categorylabel{Specifieke afspraken voor de tweede examenperiode} &
        \categorytext{
            De evaluatiemethode(n) voor examenkans 2 wijkt af van die van examenkans: 100\% van punten op schriftelijk examen.
        } \\
        \bottomrule
    \end{tabular}
\end{center}

\end{document}