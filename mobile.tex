\documentclass{studiewijzer}

\studiewijzer{mobile,academiejaar=2021-2022}

\begin{document}

\showheader

\begin{categorybox}{Organisatie en Begeleiding}
    \begin{category}{Organisatie van de lessen}
        Wekelijks is er een contactmoment van 2 uur.
    \end{category}

    \begin{category}{Begeleiding van de activiteiten}
        Tijdens de praktijksessies werken de studenten in team aan hun project en kunnen ze terecht bij de lector om vragen te stellen.
        Het merendeel van het werk moet gebeuren buiten de contactmomenten.
    \end{category}
\end{categorybox}

\begin{categorybox}{Evaluatie}
    \begin{category}{Evaluatievorm}
        Project
    \end{category}

    \begin{category}{Puntenverdeling evaluatieactiviteiten}
        Studenten moeten in team een (door de lector opgelegd) groot project uitwerken.
        Tijdens het mondeling examen moet het project verdedigd worden.
    \end{category}

    \begin{category}{Opzet van de evaluatieactiviteit}
        \begin{items}
            \item 100\% opdracht met mondelinge toelichting tijdens de examenzittijd.
            \item Studenten worden beoordeeld als team.
            \item Het is de verantwoordelijkheid van de teamleden om de goede functionering van het team te bekomen.
            \item De nadruk ligt op de correcte werking van alle benodigde functionaliteit.
        \end{items}
    \end{category}

    \begin{category}{Beoogde competenties}
        \begin{items}
            \item Werken in teamverband.
            \item Gebruik van een versiebeheersysteem.
            \item Zelfstandig met nieuwe technologie\"en leren werken.
            \item Mobiele apps ontwikkelen.
        \end{items}
    \end{category}

    \begin{category}{Tweede examenperiode}
        Voor tweede zit is er tevens een uit te werken project.
        Details zijn studentafhankelijk en worden besproken in overleg met de lector.
    \end{category}

    \begin{category}{Verwachtingen en afspraken}
        \begin{items}
            \item Het project moet te allen tijde beschikbaar gesteld worden via een versiebeheersysteem (Git) conform de afspraken (zie Toledo).
            \item Teams werken afzonderlijk. Code kan niet gedeeld worden tussen teams.
        \end{items}
    \end{category}

    \begin{category}{Deadlines}
        De deadline valt op het einde van het semester. Exacte datums worden via Toledo meegedeeld.
    \end{category}
\end{categorybox}

\end{document}
